\section{Desenvolvimento do Aplicativo}

\subsection{Tecnologias Utilizadas}

\subsubsection{Front-end}
O aplicativo FutSteps foi desenvolvido utilizando um stack moderno de tecnologias front-end, priorizando performance, manutenibilidade e experiência do usuário:

\begin{itemize}
    \item \textbf{React Native (v0.79.3)}: Framework principal para desenvolvimento mobile multiplataforma, permitindo o desenvolvimento de aplicativos nativos para iOS e Android com uma única base de código JavaScript/TypeScript.
    
    \item \textbf{Expo (v53.0.11)}: Plataforma de desenvolvimento que simplifica o processo de criação de aplicativos React Native, oferecendo ferramentas de desenvolvimento, build e deploy otimizadas.
    
    \item \textbf{TypeScript (v5.8.3)}: Linguagem de programação que adiciona tipagem estática ao JavaScript, proporcionando maior segurança de tipos, melhor IntelliSense e detecção de erros em tempo de desenvolvimento.
    
    \item \textbf{Expo Router (v5.1.0)}: Sistema de roteamento baseado em arquivos (file-based routing) que simplifica a navegação entre telas, oferecendo uma experiência similar ao Next.js para React Native.
    
    \item \textbf{NativeWind (v4.1.23)}: Biblioteca que permite o uso do Tailwind CSS no React Native, oferecendo uma abordagem utilitária para estilização com classes CSS.
    
    \item \textbf{Tailwind CSS (v3.4.17)}: Framework CSS utilitário que permite estilização rápida e consistente através de classes predefinidas, facilitando a criação de interfaces responsivas.
    
    \item \textbf{React Navigation (v7.1.9)}: Biblioteca oficial para navegação entre telas do React Native, oferecendo navegação por abas, stack e drawer.
    
    \item \textbf{Expo Vector Icons (v14.1.0)}: Biblioteca de ícones vetoriais que inclui múltiplas famílias de ícones (FontAwesome, Ionicons, etc.) otimizadas para React Native.
    
    \item \textbf{React Native Reanimated (v3.17.4)}: Biblioteca para animações fluidas e performáticas, utilizando a thread nativa para cálculos de animação.
    
    \item \textbf{React Native SVG (v15.11.2)}: Biblioteca para renderização de gráficos SVG no React Native, utilizada para criar o campo de futebol interativo.
\end{itemize}

\subsubsection{Back-end}
\textbf{Nota}: O back-end não foi desenvolvido para este projeto. Em vez disso, foram implementadas as seguintes práticas para simular funcionalidades de persistência e dados:

\begin{itemize}
    \item \textbf{AsyncStorage (v2.1.2)}: Sistema de armazenamento local assíncrono para persistência de dados de usuário no dispositivo, utilizado principalmente para autenticação e preferências do usuário.
    
    \item \textbf{Dados Mockados}: Estruturas de dados estáticas que simulam respostas de API, incluindo dados de atletas, clubes, estatísticas, treinos e lesões, permitindo o desenvolvimento e teste da interface sem dependência de um servidor.
\end{itemize}

\subsubsection{Ferramentas de Desenvolvimento e Qualidade}
\begin{itemize}
    \item \textbf{Jest (v29.2.1)}: Framework de testes unitários que oferece suporte completo para testes de componentes React Native, incluindo mocks e assertions avançadas.
    
    \item \textbf{Testing Library (v13.2.0)}: Biblioteca para testes de componentes React que prioriza testes baseados no comportamento do usuário, facilitando a criação de testes mais robustos e manuteníveis.
    
    \item \textbf{Biome (v1.9.4)}: Linter e formatter de código moderno que substitui o ESLint e Prettier, oferecendo formatação automática e detecção de problemas de código.
    
    \item \textbf{EAS Build}: Sistema de build da Expo para produção, permitindo a criação de builds nativos para iOS e Android através da nuvem, sem necessidade de configuração local de SDKs.
\end{itemize}

\subsection{Estrutura do Código Desenvolvido}

\subsubsection{Arquitetura do Projeto}
O projeto segue uma arquitetura modular e escalável, organizada em camadas bem definidas:

\begin{verbatim}
src/
├── app/                    # Camada de roteamento e telas principais
│   ├── (auth)/            # Grupo de rotas para autenticação
│   │   ├── _layout.tsx    # Layout específico para autenticação
│   │   ├── index.tsx      # Tela inicial de auth (redirecionamento)
│   │   ├── login.tsx      # Tela de login
│   │   └── register.tsx   # Tela de registro
│   ├── (tabs)/            # Grupo de rotas principais com navegação por abas
│   │   ├── _layout.tsx    # Layout com TabBar customizada
│   │   ├── home.tsx       # Tela principal com campo de futebol
│   │   ├── management.tsx # Tela de gerenciamento
│   │   └── statistics.tsx # Tela de estatísticas
│   ├── (management)/      # Grupo de rotas para funcionalidades específicas
│   │   ├── performance.tsx    # Dados de performance
│   │   ├── physical-data.tsx  # Dados físicos
│   │   └── training.tsx       # Dados de treino
│   └── _layout.tsx        # Layout raiz com providers globais
├── components/            # Camada de componentes reutilizáveis
│   ├── athlete/          # Componentes específicos de atletas
│   ├── auth/             # Componentes de autenticação
│   ├── club/             # Componentes de clubes
│   ├── common/           # Componentes comuns (Button, Input, etc.)
│   ├── futebol/          # Componentes relacionados ao futebol
│   ├── home/             # Componentes da tela inicial
│   ├── layout/           # Componentes de layout (TabBar, etc.)
│   └── physical-data/    # Componentes de dados físicos
├── lib/                  # Camada de utilitários e configurações
│   ├── storage.ts        # Funções de AsyncStorage
│   └── utils/            # Utilitários gerais
├── mocks/                # Camada de dados simulados
├── styles/               # Camada de estilos globais
├── tests/                # Camada de testes automatizados
└── types/                # Camada de definições de tipos TypeScript
\end{verbatim}

\subsubsection{Padrões de Desenvolvimento Implementados}

O código segue rigorosos padrões de desenvolvimento modernos:

\begin{itemize}
    \item \textbf{Componentização Modular}: Cada componente tem uma responsabilidade específica e é reutilizável em diferentes contextos. Por exemplo, o componente \texttt{Field} é responsável apenas pela renderização do campo de futebol, enquanto \texttt{PlayerTable} gerencia a exibição da tabela de jogadores.
    
    \item \textbf{TypeScript Rigoroso}: Tipagem forte em toda a aplicação, incluindo interfaces para props de componentes, tipos para dados mockados e definições de funções, garantindo maior segurança e manutenibilidade.
    
    \item \textbf{Tailwind CSS Utilitário}: Estilização baseada em classes utilitárias que facilita a manutenção e garante consistência visual. O projeto inclui customizações específicas de cores e fontes.
    
    \item \textbf{Separação de Responsabilidades}: Cada arquivo tem uma função específica - componentes apenas renderizam, hooks gerenciam estado, utilitários processam dados, e mocks fornecem dados de teste.
    
    \item \textbf{Testes Automatizados}: Cobertura de testes para componentes críticos, incluindo testes de acessibilidade, validação, performance e segurança.
\end{itemize}

\subsubsection{Configuração do Projeto}

O projeto utiliza configurações avançadas para otimizar o desenvolvimento:

\begin{verbatim}
// app.json - Configuração principal do Expo
{
  "expo": {
    "name": "futsteps",
    "slug": "futsteps",
    "version": "1.0.0",
    "orientation": "portrait",
    "userInterfaceStyle": "automatic",
    "newArchEnabled": true,  // Nova arquitetura do React Native
    "experiments": {
      "typedRoutes": true    // Rotas tipadas
    }
  }
}
\end{verbatim}

\subsection{Integração com Banco de Dados}

\subsubsection{AsyncStorage para Autenticação}

O sistema de autenticação foi implementado utilizando AsyncStorage para persistência local, simulando um sistema de banco de dados local:

\begin{verbatim}
// src/lib/storage.ts - Sistema de armazenamento local
import AsyncStorage from '@react-native-async-storage/async-storage'

export interface User {
  name: string
  email: string
  password: string
}

const USERS_KEY = 'users'

// Função para recuperar todos os usuários
export async function getUsers(): Promise<User[]> {
  const data = await AsyncStorage.getItem(USERS_KEY)
  if (!data) return []
  try {
    return JSON.parse(data)
  } catch {
    return []
  }
}

// Função para salvar um novo usuário
export async function saveUser(user: User): Promise<void> {
  const users = await getUsers()
  users.push(user)
  await AsyncStorage.setItem(USERS_KEY, JSON.stringify(users))
}

// Função para autenticar usuário
export async function findUserByEmailAndPassword(
  email: string,
  password: string
): Promise<User | undefined> {
  const users = await getUsers()
  return users.find(u => u.email === email && u.password === password)
}
\end{verbatim}

\subsubsection{Implementação do Sistema de Login}

O processo de autenticação é implementado de forma segura e responsiva:

\begin{verbatim}
// src/app/(auth)/login.tsx - Lógica de autenticação
async function handleLogin(data: { email: string; password: string }) {
  const users = await getUsers()
  const user = users.find(u => u.email === data.email)
  
  if (!user) {
    showToast('Usuário não encontrado!', 'error')
    return
  }
  
  if (user.password !== data.password) {
    showToast('Senha incorreta!', 'error')
    return
  }
  
  showToast('Login realizado com sucesso!', 'success')
  setTimeout(() => router.replace('/(tabs)/home'), 1000)
}
\end{verbatim}

\subsubsection{Sistema de Dados Mockados}

Para simular dados de atletas, clubes e estatísticas, foi criado um sistema robusto de dados mockados:

\begin{verbatim}
// src/mocks/mock-training.ts - Estrutura de dados de treino
export type Player = {
  id: number
  number: number
  name: string
  age: number
  position: string
  teamType: 'titular' | 'reserva'
}

export const mockTraining = {
  id: 1,
  formationTitular: '4-3-3',
  formationReserve: '3-4-3',
  players: [
    // Titulares (4-3-3)
    {
      id: 1,
      number: 21,
      name: 'Frankie de Jong',
      age: 26,
      position: 'CM',
      teamType: 'titular',
    },
    // ... outros jogadores
  ] as Player[],
}
\end{verbatim}

\subsubsection{Componentes Específicos do Futebol}

O aplicativo inclui componentes especializados para visualização de dados esportivos:

\begin{verbatim}
// src/components/futebol/field.tsx - Campo de futebol interativo
export function Field({
  players,
  formationTitular,
  formationReserve,
  onPlayerPress,
  onPlayerSeeMore,
}: FieldProps) {
  const [selectedPlayer, setSelectedPlayer] = useState<Player | null>(null)
  
  // Mapeamento de posições para diferentes formações
  const positionMaps: { [formation: string]: PositionMap } = {
    '4-3-3': {
      GK: { x: 0.08, y: 0.5 },
      LB: { x: 0.2, y: 0.15 },
      // ... outras posições
    },
    '3-4-3': {
      GK: { x: 0.92, y: 0.5 },
      CB: { x: 0.8, y: 0.25 },
      // ... outras posições
    },
  }
  
  return (
    <View>
      <Svg width={FIELD_WIDTH} height={FIELD_HEIGHT}>
        {/* Renderização do campo com SVG */}
        {/* Posicionamento dinâmico dos jogadores */}
      </Svg>
    </View>
  )
}
\end{verbatim}

\subsubsection{Sistema de Navegação Customizada}

O aplicativo implementa uma TabBar customizada com animações fluidas:

\begin{verbatim}
// src/components/layout/tab-bar.tsx - TabBar customizada
export default function TabBar({
  state,
  descriptors,
  navigation,
}: BottomTabBarProps) {
  const primaryColor = '#FFCC26'
  const secondaryColor = '#876803'
  
  return (
    <View className="flex bottom-3 flex-row bg-gr-1100 items-center 
                    justify-between gap-1 mx-3 px-6 rounded-3xl 
                    shadow-sm shadow-yl-100">
      {state.routes.map((route, index) => {
        const isFocused = state.index === index
        return (
          <TabBarButton
            key={route.name}
            isFocused={isFocused}
            color={isFocused ? primaryColor : secondaryColor}
            // ... outras props
          />
        )
      })}
    </View>
  )
}
\end{verbatim}

\subsubsection{Sistema de Testes Automatizados}

O projeto inclui uma suíte completa de testes:

\begin{verbatim}
// src/tests/components/form-input.test.tsx - Testes de componentes
describe('FormInput', () => {
  it('validates email input correctly', () => {
    const { getByPlaceholderText } = render(
      <FormInput
        type="email"
        placeholder="Email input"
        onChangeText={mockOnChangeText}
        validationRules={[
          {
            validate: value => /^[^\s@]+@[^\s@]+\.[^\s@]+$/.test(value),
            message: 'Must be a valid email',
          },
        ]}
        onValidationChange={mockOnValidationChange}
      />
    )
    
    const input = getByPlaceholderText('Email input')
    
    // Test valid email
    fireEvent.changeText(input, 'test@example.com')
    expect(mockOnValidationChange).toHaveBeenCalledWith(true)
    
    // Test invalid email
    fireEvent.changeText(input, 'invalid-email')
    expect(mockOnValidationChange).toHaveBeenCalledWith(false)
  })
  
  it('prevents SQL injection attempts', () => {
    // Testes de segurança contra injeção SQL
    fireEvent.changeText(input, "'; DROP TABLE users; --")
    expect(mockOnValidationChange).toHaveBeenCalledWith(false)
  })
})
\end{verbatim}

\subsection{Repositório no GitHub}

O código fonte completo do projeto está disponível no repositório público:
\begin{center}
\textbf{https://github.com/seu-usuario/futsteps-mobile}
\end{center}

O repositório inclui:
\begin{itemize}
    \item Código fonte completo com TypeScript
    \item Configurações de build e deploy
    \item Documentação técnica detalhada
    \item Scripts de teste automatizados
    \item Configurações de linting e formatação
\end{itemize}

\subsection{Demo do Aplicativo}

Vídeos demonstrativos do aplicativo em execução estão disponíveis nos seguintes links:

\begin{itemize}
    \item \textbf{Ambiente Web}: [Link para vídeo demo web] - Demonstração da versão web responsiva
    \item \textbf{Android}: [Link para vídeo demo Android] - Funcionalidades nativas no Android
    \item \textbf{iOS}: [Link para vídeo demo iOS] - Funcionalidades nativas no iOS
\end{itemize}

\textbf{Nota}: Os links para os vídeos demo devem ser adicionados conforme disponibilidade. Os vídeos demonstram:
\begin{itemize}
    \item Fluxo completo de autenticação
    \item Navegação entre telas
    \item Campo de futebol interativo
    \item Visualização de estatísticas
    \item Responsividade em diferentes tamanhos de tela
\end{itemize}
